\documentclass{scrbook}

\usepackage[ngerman]{babel}
\usepackage[T1]{fontenc}
\usepackage[utf8]{inputenc}

\usepackage{hyperref}
\usepackage{csquotes}
\MakeAutoQuote{„}{“}
\usepackage{cleveref}

\usepackage[scale=0.95]{tgpagella}
\usepackage[scale=0.92]{tgheros}
\usepackage[scaled=0.83]{beramono}
\usepackage{mathpazo}

\setlength{\parindent}{0cm}
\setlength{\parskip}{0.5ex}

\usepackage[style=apa]{biblatex}

\usepackage{makeidx}
\makeindex

\usepackage{tabu}
\usepackage{booktabs}

\usepackage{todonotes}

\begin{document}

\frontmatter

\begin{titlepage}

  \begin{center}
    \Huge

    Technische Universität Dresden

    Fachrichtung Zukunfswissenschaften

    \bigskip

    \LARGE

    Institut für Höhere Kognition

    \vfill

    \huge

    \textbf{Über das Sein und Nichtsein von\\ Werden und Vergehen}

    \LARGE

    \vfill

    Diplomarbeit \\
    zur Erlangung des ersten akademischen Grades

    \bigskip

    \textbf{Diplomakademikerin}

  \end{center}

  \vfill

  \Large

  vorgelegt von

  \vspace*{2\bigskipamount}

  \begin{tabular}{@{}lp{4cm}@{\qquad}rl@{}}
    Name:       & Lustig     & Vorname: & Lutetia\\[2.0ex]
    geboren am: & 17.03.1976 & in:      & Duckburg
  \end{tabular}

  \vspace*{4\bigskipamount}

  Tag der Einreichung: 29.02.2016

  \vspace*{2\bigskipamount}

  Betreuer: Prof. Dr. nihil. Siegfried von Gestern

\end{titlepage}

\newpage

\hbox{}\vfill

\noindent
Copyright \copyright\  2016 Lutetia Lustig\\
This work is licensed under the Creative Commons Attribution-ShareAlike 4.0
International License. To view a copy of this license, visit
\url{http://creativecommons.org/licenses/by-sa/4.0/deed.en_US}.

\chapter*{Vorwort}

In unserer virtuellen Zeit ist es wichtiger denn je, sich des Unterschieds
zwischen dem Hier und dem Jetzt klar zu werden.  Eine einfacher Verwischung
dieser beiden doch sehr unterschiedlichen Begriffe kann dafür sorgen, dass wir
uns in unserem Denke und Handeln nicht mehr leiten lassen von den Motivation
unser Intuition, sondern von der Versprechungen der externen Welt.  Ein Teil
dieser Arbeit soll es sein, sich diesen Begrifflichkeiten auf wissenschaftlichem
Wege zu nähern und so auch für die Außenstehende und den Außenstehenden klar zu
machen, was die Kerngedanken in der Unterscheidung von Hier und Jetzt sind.

Die Arbeit lässt sich grob in drei Teile gliedern.  In \Cref{cha:eins} befassen
wir uns mit den Grundbegrifflichkeiten der modernen Wesenslehre.  In
\ref{cha:zwei} ziehen wir erst Verbindungen zu Beobachtungen aus der realen
Welt, und in \Cref{cha:drei} schließlich führen wir die Ergebnisse der
vorherigen Abschnitte in einer kontroversen Argumentationsstruktur zusammen.

\cleardoublepage

\tableofcontents

\mainmatter

\chapter{Eins, \dots}
\label{cha:eins}

Es geht los, seien Sie gespannt!

\section{Der erste Streich!}
\label{sec:der-erste-streich}

Wir beginnen mit einer einfach Betrachtung klassischer Beispiele.  Dafür
verweisen wir zuerst eloquent auf die folgende Tabelle verweisen:

\begin{center}
  \begin{tabu}{XX}
    \toprule
    Begriff & Bedeutung \\\midrule
    Sein des Wesens und des Geistes & Die Quintessenz aller Dinge \\
    Nichtsein als Verständnis des Nichts & Das Antonym des Wesean \\
    \bottomrule
  \end{tabu}
\end{center}

\section{Und der zweite folgt sogleich!}
\label{sec:und-der-zweite}

Diese Begrifflichkeiten entziehen sich nicht einer bestimmten Mystik, die sie
seit ihrer Entstehung in der Wiege der menschlichen Gedankenwelt genießen.  Eine
ebenso wichtige Begrifflichkeit mit nicht weniger Mystik sind die
\emph{Primzahlen}, die für diese Arbeit zwar nicht relevant sind, aber dennoch
in keiner Arbeit fehlen dürfen!  Wir zeigen die ersten 10000 Primzahlen in
\Cref{tab:prime-numbers}.

\begin{table}[tp]
  \centering
  \begin{tabu}{XXXXXXXXXX}
    \toprule
      2 &    3 &    5 &    7 &   11 &   13 &   17 &   19 &   23 &   29 \\
     31 &   37 &   41 &   43 &   47 &   53 &   59 &   61 &   67 &   71 \\
     73 &   79 &   83 &   89 &   97 &  101 &  103 &  107 &  109 &  113 \\
    127 &  131 &  137 &  139 &  149 &  151 &  157 &  163 &  167 &  173 \\
    179 &  181 &  191 &  193 &  197 &  199 &  211 &  223 &  227 &  229 \\
    233 &  239 &  241 &  251 &  257 &  263 &  269 &  271 &  277 &  281 \\
    283 &  293 &  307 &  311 &  313 &  317 &  331 &  337 &  347 &  349 \\
    353 &  359 &  367 &  373 &  379 &  383 &  389 &  397 &  401 &  409 \\
    419 &  421 &  431 &  433 &  439 &  443 &  449 &  457 &  461 &  463 \\
    467 &  479 &  487 &  491 &  499 &  503 &  509 &  521 &  523 &  541 \\
    547 &  557 &  563 &  569 &  571 &  577 &  587 &  593 &  599 &  601 \\
    607 &  613 &  617 &  619 &  631 &  641 &  643 &  647 &  653 &  659 \\
    661 &  673 &  677 &  683 &  691 &  701 &  709 &  719 &  727 &  733 \\
    739 &  743 &  751 &  757 &  761 &  769 &  773 &  787 &  797 &  809 \\
    811 &  821 &  823 &  827 &  829 &  839 &  853 &  857 &  859 &  863 \\
    877 &  881 &  883 &  887 &  907 &  911 &  919 &  929 &  937 &  941 \\
    947 &  953 &  967 &  971 &  977 &  983 &  991 &  997 & 1009 & 1013 \\
   1019 & 1021 & 1031 & 1033 & 1039 & 1049 & 1051 & 1061 & 1063 & 1069 \\
   1087 & 1091 & 1093 & 1097 & 1103 & 1109 & 1117 & 1123 & 1129 & 1151 \\
   1153 & 1163 & 1171 & 1181 & 1187 & 1193 & 1201 & 1213 & 1217 & 1223 \\
   1229 & 1231 & 1237 & 1249 & 1259 & 1277 & 1279 & 1283 & 1289 & 1291 \\
   1297 & 1301 & 1303 & 1307 & 1319 & 1321 & 1327 & 1361 & 1367 & 1373 \\
   1381 & 1399 & 1409 & 1423 & 1427 & 1429 & 1433 & 1439 & 1447 & 1451 \\
   1453 & 1459 & 1471 & 1481 & 1483 & 1487 & 1489 & 1493 & 1499 & 1511 \\
   1523 & 1531 & 1543 & 1549 & 1553 & 1559 & 1567 & 1571 & 1579 & 1583 \\
   1597 & 1601 & 1607 & 1609 & 1613 & 1619 & 1621 & 1627 & 1637 & 1657 \\
   1663 & 1667 & 1669 & 1693 & 1697 & 1699 & 1709 & 1721 & 1723 & 1733 \\
   1741 & 1747 & 1753 & 1759 & 1777 & 1783 & 1787 & 1789 & 1801 & 1811 \\
   1823 & 1831 & 1847 & 1861 & 1867 & 1871 & 1873 & 1877 & 1879 & 1889 \\
   1901 & 1907 & 1913 & 1931 & 1933 & 1949 & 1951 & 1973 & 1979 & 1987 \\
   1993 & 1997 & 1999 & 2003 & 2011 & 2017 & 2027 & 2029 & 2039 & 2053 \\
   2063 & 2069 & 2081 & 2083 & 2087 & 2089 & 2099 & 2111 & 2113 & 2129 \\
   2131 & 2137 & 2141 & 2143 & 2153 & 2161 & 2179 & 2203 & 2207 & 2213 \\
   2221 & 2237 & 2239 & 2243 & 2251 & 2267 & 2269 & 2273 & 2281 & 2287 \\
   2293 & 2297 & 2309 & 2311 & 2333 & 2339 & 2341 & 2347 & 2351 & 2357 \\
   2371 & 2377 & 2381 & 2383 & 2389 & 2393 & 2399 & 2411 & 2417 & 2423 \\
   \bottomrule
  \end{tabu}
  \caption{Die ersten 10000 Primzahlen (nicht ganz \dots)}
  \label{tab:prime-numbers}
\end{table}

\section{Das War's}
\label{sec:das-wars}

Damit ist schon das wichtigste dieses Abschnittes gesagt.  Alles weitere findet
sich in \todo{Referenz}

\chapter{\dots, Zwei, \dots}
\label{cha:zwei}

\chapter{\dots, Drei!}
\label{cha:drei}

\appendix

\chapter{Dinge, die noch gesagt werden müssen}
\label{cha:dinge-die-noch}

\backmatter

\printindex

\cleardoublepage

\thispagestyle{empty}

\hbox{}\vfill

\textbf{\large ERKLÄRUNG}

\bigskip \medskip

Hiermit erkläre ich, dass ich die am heutigen Tag eingereichte Diplomarbeit zum
Thema „Über das Sein und Nichtsein von Werden und Vergehen“ unter Betreuung von
Prof.~Dr.~nihil.~Siegfried von Gestern selbstständig erarbeitet, verfasst und
Zitate kenntlich gemacht habe. Andere als die angegebenen Hilfsmittel wurden von
mir nicht benutzt.

\vspace*{5\bigskipamount}

Datum \hfill Unterschrift

\normalsize

\vspace*{2\bigskipamount}

\vfill\hbox{}

\end{document}